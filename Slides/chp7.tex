\documentclass{beamer}
\usepackage{amsmath,amssymb,latexsym,array,fancyheadings,mathdots}
\usepackage{algorithm,algorithmic}
\usepackage{hyperref}
\usepackage{color}
\usepackage{tabularx}
\usepackage[all]{xy}
\usepackage{qtree}
\usepackage{gitinfo2}

%% RCS
%\usepackage{rcs}

%% Colors
\definecolor{darkgreen}{rgb}{0,.4,0}
\definecolor{darkred}{rgb}{.5,0,0}
\definecolor{darkmagenta}{rgb}{.5,0,.5}
\definecolor{orange}{rgb}{1,.5,0}
\definecolor{lightblue}{rgb}{0.122,0.016,0.855}
\definecolor{darkocre}{rgb}{0.471,0.298,0.008}

\usetheme{default}

%% New Theorems
\newtheorem{thm}{Theorem}
\newtheorem{exm}[thm]{Example}
\newtheorem{cor}[thm]{Corollary}
\newtheorem{propo}[thm]{Proposition}
\newtheorem{lem}[thm]{Lemma}
\newtheorem{clm}[thm]{Claim}
\newtheorem{exr}[thm]{Exercise}
\newtheorem{dfn}[thm]{Definition}

%% New commands
\newcommand{\classfont}{\mathsf}
\newcommand{\ATM}{\classfont{A}_{\mathrm{TM}}}
\newcommand{\MTF}{\mathrm{MTF}}
\newcommand{\OPT}{\mathrm{OPT}}
\newcommand{\ALG}{\mathrm{ALG}}
\newcommand{\ALGNAIVE}{\mathrm{ALG}_{\text{na{\"\i}ve}}}
\newcommand{\LRU}{\mathrm{LRU}}
\newcommand{\FIFO}{\mathrm{FIFO}}
\newcommand{\FWF}{\mathrm{FWF}}
\newcommand{\LFD}{\mathrm{LFD}}
\newcommand{\true}{\mathsf{T}}
\newcommand{\false}{\mathsf{F}}
\newcommand{\also}{\wedge}
\newcommand{\lra}{\leftrightarrow}
\newcommand{\tc}{\textcolor}
\newcommand{\df}[1]{\textcolor{red}{\em #1}}
\newcommand{\highlight}[1]{\textcolor{orange}{\em #1}}
\newcommand{\hl}[1]{\textcolor{blue}{\em #1}}
\newcommand{\amp}{\texttt{\&}}
\newcommand{\hsh}{\texttt{\#}}
\newcommand{\ra}{\rightarrow}
\newcommand{\longra}{\longrightarrow}
\newcommand{\Ra}{\Rightarrow}
\newcommand{\rab}{{\rightarrow_\beta}}
\newcommand{\srab}{{\rightarrow^*_\beta}}
\newcommand{\aeq}{{=_\alpha}}
\newcommand{\order}{\mathrm{order}}
\newcommand{\rem}{\mathrm{rem}}
\newcommand{\IP}{\mathbf{IP}}
\newcommand{\PSPACE}{\mathbf{PSPACE}}
\newcommand{\thevalue}{\text{value}}
\newcommand{\pol}[1]{\mathbf{#1}}
\newcommand{\enc}{\text{Enc}}
\newcommand{\xor}{\oplus}
\newcommand{\zo}{\{0,1\}}
\newcommand{\SOPT}{S_{\mathrm{opt}}}
\newcommand{\la}{\leftarrow}
\newcommand{\myurl}[1]{\textcolor{darkgreen}{\url{#1}}}
\newcommand{\myhref}[2]{\textcolor{darkgreen}{\href{#1}{#2}}}
\newcommand{\qaccept}{q_{\mathrm{accept}}}
\newcommand{\qreject}{q_{\mathrm{reject}}}
\newcommand{\opt}{\text{\sc Opt}}
\newcommand{\tr}{\mathrm{tr}}
\newcommand{\csanky}{p^{\textsc{csanky}}}
\newcommand{\berk}{p^{\textsc{berk}}}

%% Algorithms package customization
\renewcommand{\algorithmicrequire}{\textbf{Pre-condition:}} 
\renewcommand{\algorithmicensure}{\textbf{Post-condition:}} 
\algsetup{indent=3em}

\input{prooftree}

%% including/excluding pauses
\newcommand{\ifpause}{\iftrue} % for including pauses
%\newcommand{\ifpause}{\iffalse} % for excluding pauses

%% 2nd or 3rd edition
\newif\ifthird
\thirdtrue
%\thirdfalse

%disables usefoottemplate
\setbeamertemplate{navigation symbols}{}
%\setbeamertemplate{footline}% 
%{\strut\quad\tiny 
%\begin{minipage}{3cm}
%Cryptography - Michael Soltys
%\today\ {\tt v\RCSRevision}
%\end{minipage}\hfill
%\insertsection\
%- \insertframenumber/\inserttotalframenumber\quad\strut}

\newcommand{\mytitle}{Linera Algebra / Parallel}
\newcommand{\mychpnr}{7}
%% Title page contents
\title{Intro to Analysis of Algorithms \\ \mytitle \\  Chapter \mychpnr}
\author{Michael Soltys}
\date{\textcolor{darkgreen}{\tiny\tt 
[Ed: 4th, last updated: \today]}}
\institute{CSU Channel Islands}

\setbeamertemplate{footline}{
  \colorbox{white}{\color{black}\tt
     \begin{tabularx}{0.97\textwidth}{XXX}
          IAA Chp \mychpnr\ - Michael Soltys \copyright & 
          \hfill\today\ (Ed: 4th)
					\hfill\phantom{.} & 
          \hfill\insertsection\ - \insertframenumber/\inserttotalframenumber \\
      \end{tabularx}}}

\begin{document}

\mode<presentation>
{
}

\parskip 8pt

\section{Introduction}

\begin{frame}
\titlepage
\end{frame}


\section{Introduction}

%\addfootbox{\tiny}

\section{Gaussian Elimination}

\begin{frame}
\frametitle{Row-echelon form}

$$
\left[\begin{array}{cccccccc}
1 & *\ldots * & * & *\ldots * & * & *\ldots * & * & \\
  &           & 1 & *\ldots * & * & *\ldots * & * & \\
  &  \ddots   &   &           & 1 & *\ldots * & * & \\
  &           & 0 &           &   &           & 1 & \ldots \\
  &           &   &  \ddots   &   &           & \vdots & \ddots
\end{array}\right]
$$
\end{frame}

\begin{frame}
\frametitle{Elementary matrices}

one of the following three forms:
\begin{align*}
& I+aT_{ij} \quad i\neq j        \tag{elementary of type 1} \\
& I+T_{ij}+T_{ji}-T_{ii}-T_{jj}  \tag{elementary of type 2} \\
& I+(c-1)T_{ii} \quad c\neq 0    \tag{elementary of type 3}
\end{align*} 
\end{frame}

\begin{frame}
Gaussian Elimination is a divide and conquer algorithm,
with a recursive call to smaller matrices.

If $A$ is a $1\times m$ matrix, $A=[a_{11} a_{12} \ldots a_{1m}]$,
then:
$$
GE(A)=\begin{cases}
[1/a_{1i}] & \text{where $i=\min\{1,2,\ldots,m\}$ such that
                   $a_{i1}\neq 0$} \\
[1]        & \text{if $a_{11}=a_{12}=\cdots=a_{1m}=0$}
\end{cases}
$$

Suppose now that $n>1$.  If $A=0$, let $GE(A)=I$.  Otherwise, let:
$$
GE(A)=\left[\begin{array}{cc}
1 & 0 \\
0 & GE((EA)[1|1])
\end{array}\right]E
$$
where $E$ is a product of at most $n+1$ elementary matrices.
Note that $C[i|j]$ denotes the matrix $C$ with row $i$ and $j$.
\end{frame}

\begin{frame}
\begin{algorithmic}[1]
\IF {$n=1$}
	\IF {$a_{11}=a_{12}=\cdots=a_{1m}=0$}
		\RETURN $[1]$
	\ELSE
		\RETURN $[1/a_{1\ell}]$ where $\ell=\min_{i\in [n]}\{a_{1i}\neq 0\}$
	\ENDIF
\ELSE
	\IF {$A=0$}
		\RETURN $I$
	\ELSE
		\IF {first column of $A$ is zero}
			\STATE Compute $E$ as in Case 1.
		\ELSE
			\STATE Compute $E$ as in Case 2.
		\ENDIF
		\RETURN $\left[\begin{array}{cc}
						1 & 0 \\
						0 & GE((EA)[1|1])
						\end{array}\right]E$
	\ENDIF
\ENDIF 
\end{algorithmic}

\end{frame}

\section{Gram-Schmidt}

\begin{frame}
\frametitle{Gram-Schmidt}

\begin{algorithmic}[1] 
\REQUIRE $\{v_1,\ldots,v_n\}$ a basis for $\mathbb{R}^n$
\STATE $v_1^*\longleftarrow v_1$ 
\FOR{$i=2,3,\ldots,n$} 
     \FOR{$j=1,2,\ldots,(i-1)$} 
          \STATE $\mu_{ij}\longleftarrow (v_i\cdot v_j^*)/\|v_j^*\|^2$ 
     \ENDFOR
     \STATE $v_i^*\longleftarrow v_i-\sum_{j=1}^{i-1}\mu_{ij}v_j^*$
\ENDFOR
\ENSURE $\{v_1^*,\ldots,v_n^*\}$ an orthogonal basis for $\mathbb{R}^n$
\end{algorithmic}
\end{frame}

\section{Guass lattice reduction}

\begin{frame}
\frametitle{Gauss lattice reduction}

\begin{algorithmic}[1]
\REQUIRE $\{v_1,v_2\}$ are linearly independent in $\mathbb{R}^2$
\LOOP
	\IF{$\|v_2\|<\|v_1\|$}
		\STATE swap $v_1$ and $v_2$
	\ENDIF
	\STATE $m\longleftarrow\lfloor v_1\cdot v_2/\|v_1\|^2\rceil$
	       (note that $\lfloor x\rceil=\lfloor x+1/2\rfloor$)
	\IF{$m=0$}
		\RETURN $v_1,v_2$
	\ELSE
		\STATE $v_2\longleftarrow v_2-mv_1$
	\ENDIF
\ENDLOOP
\end{algorithmic}
\end{frame}

\section{Csanky's algorithm}

\begin{frame}
\frametitle{Csanky}

Given a matrix $A$, its {\em trace} is defined as
the sum of the diagonal entries, i.e., $\tr(A)=\sum_ia_{ii}$. Using
traces we can compute the 
{\em Newton's symmetric polynomials} which
are defined as follows: $s_0=1$, and for $1\le k\le n$, by:
$$
s_k=\frac{1}{k}\sum_{i=1}^{k}(-1)^{i-1}s_{k-i}\tr(A^i).
$$
Then, it turns out that $p_A(x)=s_0x^n-s_1x^{n-1}+s_2x^{n-2}-\cdots\pm s_nx^0$,
that is, Newton's symmetric polynomials compute the coefficients of
the characteristic polynomial, $p_A(x)=\det(xI-A)$.
\end{frame}

\begin{frame}
$$
\left(\begin{array}{c} s_1\\s_2\\\vdots\\s_n \end{array}\right),
\quad
\left(\setlength{\extrarowheight}{8pt}\begin{array}{llll}
0                   & 0                   & 0                 & \ldots \\
\frac{1}{2}\tr(A)   & 0                   & 0                 & \ldots \\
\frac{1}{3}\tr(A^2) & \frac{1}{3}\tr(A)   & 0                 & \ldots \\
\frac{1}{4}\tr(A^3) & \frac{1}{4}\tr(A^2) & \frac{1}{4}\tr(A) & \ldots \\
\vdots              & \vdots              & \vdots            & \ddots
\end{array}\right),
\quad
\left(\setlength{\extrarowheight}{8pt}\begin{array}{l} 
\tr(A) \\
\frac{1}{2}\tr(A^2) \\
\vdots \\
\frac{1}{n}\tr(A^n)
\end{array}\right)
$$

\end{frame}

\section{Berkowitz's algorithm}

\begin{frame}
\frametitle{Berkowitz}
Berkowitz's algorithm is also Divide and Conquer, and it computes
the characteristic polynomial of $A$ from the characteristic
polynomial of its {\em principal minor},
i.e., the matrix $M$ obtained from deleting the first row and column
of $A$:
$$
A=\left(\begin{array}{cc}
a_{11} & R \\
S      & M
\end{array}\right),
$$
\end{frame}

\begin{frame}
$R$ is an $1\times(n-1)$ row matrix and $S$ is a $(n-1)\times 1$
column matrix and $M$ is $(n-1)\times(n-1)$.  Let $p(x)$ and $q(x)$ be
the characteristic polynomials of $A$ and $M$ respectively.  Suppose
that the coefficients of $p$ form the column vector: 
$$
p=\left(\begin{array}{cccc}
p_n&p_{n-1}&\ldots&p_0\end{array}\right)^t,
$$
where $p_i$ is the coefficient of $x^i$ in $\det(xI-A)$, and similarly
for $q$.  Then:
$$
p=C_1q,
$$
where $C_1$ is an $(n+1)\times n$ {\em
Toeplitz} lower triangular matrix
(Toeplitz means that the values on each diagonal are constant)
\end{frame}

\begin{frame}
where the entries in the first column are defined as follows:
$c_{i1}=1$ if $i=1$, $c_{i1}=-a_{11}$ if $i=2$, and
$c_{i1}=-(RM^{i-3}S)$ if $i\geq 3$.
Berkowitz's algorithm consists in repeating this for $q$, and
continuing so that $p$ is expressed as a product of matrices.  Thus:
$$
\berk_A=C_1C_2\cdots C_n,
$$
where $C_i$ is an $(n+2-i)\times (n+1-i)$ Toeplitz matrix defined as
above except $A$ is replaced by its $i$-th principal
sub-matrix.  Note that $C_n=(1 \ \  -a_{nn})^t$.
\end{frame}

\end{document}
