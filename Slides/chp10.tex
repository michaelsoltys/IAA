\documentclass{beamer}
\usepackage{amsmath,amssymb,latexsym,array,fancyheadings,mathdots}
\usepackage{algorithm,algorithmic}
\usepackage{hyperref}
\usepackage{color}
\usepackage{tabularx}
\usepackage[all]{xy}
\usepackage{qtree}
\usepackage{gitinfo2}

%% RCS
%\usepackage{rcs}

%% Colors
\definecolor{darkgreen}{rgb}{0,.4,0}
\definecolor{darkred}{rgb}{.5,0,0}
\definecolor{darkmagenta}{rgb}{.5,0,.5}
\definecolor{orange}{rgb}{1,.5,0}
\definecolor{lightblue}{rgb}{0.122,0.016,0.855}
\definecolor{darkocre}{rgb}{0.471,0.298,0.008}

\usetheme{default}

%% New Theorems
\newtheorem{thm}{Theorem}
\newtheorem{exm}[thm]{Example}
\newtheorem{cor}[thm]{Corollary}
\newtheorem{propo}[thm]{Proposition}
\newtheorem{lem}[thm]{Lemma}
\newtheorem{clm}[thm]{Claim}
\newtheorem{exr}[thm]{Exercise}
\newtheorem{dfn}[thm]{Definition}

%% New commands
\newcommand{\classfont}{\mathsf}
\newcommand{\ATM}{\classfont{A}_{\mathrm{TM}}}
\newcommand{\MTF}{\mathrm{MTF}}
\newcommand{\OPT}{\mathrm{OPT}}
\newcommand{\ALG}{\mathrm{ALG}}
\newcommand{\ALGNAIVE}{\mathrm{ALG}_{\text{na{\"\i}ve}}}
\newcommand{\LRU}{\mathrm{LRU}}
\newcommand{\FIFO}{\mathrm{FIFO}}
\newcommand{\FWF}{\mathrm{FWF}}
\newcommand{\LFD}{\mathrm{LFD}}
\newcommand{\true}{\mathsf{T}}
\newcommand{\false}{\mathsf{F}}
\newcommand{\also}{\wedge}
\newcommand{\lra}{\leftrightarrow}
\newcommand{\tc}{\textcolor}
\newcommand{\df}[1]{\textcolor{red}{\em #1}}
\newcommand{\highlight}[1]{\textcolor{orange}{\em #1}}
\newcommand{\hl}[1]{\textcolor{blue}{\em #1}}
\newcommand{\amp}{\texttt{\&}}
\newcommand{\hsh}{\texttt{\#}}
\newcommand{\ra}{\rightarrow}
\newcommand{\longra}{\longrightarrow}
\newcommand{\Ra}{\Rightarrow}
\newcommand{\rab}{{\rightarrow_\beta}}
\newcommand{\srab}{{\rightarrow^*_\beta}}
\newcommand{\aeq}{{=_\alpha}}
\newcommand{\order}{\mathrm{order}}
\newcommand{\rem}{\mathrm{rem}}
\newcommand{\IP}{\mathbf{IP}}
\newcommand{\PSPACE}{\mathbf{PSPACE}}
\newcommand{\thevalue}{\text{value}}
\newcommand{\pol}[1]{\mathbf{#1}}
\newcommand{\enc}{\text{Enc}}
\newcommand{\xor}{\oplus}
\newcommand{\zo}{\{0,1\}}
\newcommand{\SOPT}{S_{\mathrm{opt}}}
\newcommand{\la}{\leftarrow}
\newcommand{\myurl}[1]{\textcolor{darkgreen}{\url{#1}}}
\newcommand{\myhref}[2]{\textcolor{darkgreen}{\href{#1}{#2}}}
\newcommand{\qaccept}{q_{\mathrm{accept}}}
\newcommand{\qreject}{q_{\mathrm{reject}}}
\newcommand{\opt}{\text{\sc Opt}}
\newcommand{\tr}{\mathrm{tr}}
\newcommand{\csanky}{p^{\textsc{csanky}}}
\newcommand{\berk}{p^{\textsc{berk}}}

%% Algorithms package customization
\renewcommand{\algorithmicrequire}{\textbf{Pre-condition:}} 
\renewcommand{\algorithmicensure}{\textbf{Post-condition:}} 
\algsetup{indent=3em}

\input{prooftree}

%% including/excluding pauses
\newcommand{\ifpause}{\iftrue} % for including pauses
%\newcommand{\ifpause}{\iffalse} % for excluding pauses

%% 2nd or 3rd edition
\newif\ifthird
\thirdtrue
%\thirdfalse

%disables usefoottemplate
\setbeamertemplate{navigation symbols}{}
%\setbeamertemplate{footline}% 
%{\strut\quad\tiny 
%\begin{minipage}{3cm}
%Cryptography - Michael Soltys
%\today\ {\tt v\RCSRevision}
%\end{minipage}\hfill
%\insertsection\
%- \insertframenumber/\inserttotalframenumber\quad\strut}

\newcommand{\mytitle}{Mathematical Foundations}
\newcommand{\mychpnr}{9}
%% Title page contents
\title{Intro to Analysis of Algorithms \\ \mytitle \\  Chapter \mychpnr}
\author{Michael Soltys}
\date{\textcolor{darkgreen}{\tiny\tt 
[ {\bf Git} Date:\gitAuthorDate\ 
Hash:\gitAbbrevHash\ 
Ed:\ifthird
3rd
\else
2nd
\fi]}}
\institute{CSU Channel Islands}

\setbeamertemplate{footline}{
  \colorbox{white}{\color{black}\tt
     \begin{tabularx}{0.97\textwidth}{XXX}
          IAA Chp \mychpnr\ - Michael Soltys \copyright & 
          \hfill\today\ (\gitAbbrevHash; \ifthird ed3\else ed2\fi)
					\hfill\phantom{.} & 
          \hfill\insertsection\ - \insertframenumber/\inserttotalframenumber \\
      \end{tabularx}}}

\begin{document}

\mode<presentation>
{
}

\parskip 8pt

\section{Introduction}

\begin{frame}
\titlepage
\end{frame}


\section{Number Theory}

\begin{frame}
\frametitle{Number theory}

$\mathbb{Z}=\{\ldots,-3,-2,-1,0,1,2,3,\ldots\}$

$\mathbb{N}=\{0,1,2,\ldots\}$  

We say that $x$ \df{divides}
$y$, and write $x|y$ if $y=qx$.  

If $x|y$ we
say that $x$ is \df{divisor} (also \df{factor}) of $y$.

$x|y$ iff
$y=\text{div}(x,y)\cdot x$.  

We say that a number $p$ is \df{prime} if its only divisors are itself
and~1.
\end{frame}

\begin{frame}

{\bf Claim:}
If $p$ is a prime, and $p|a_1a_2\ldots a_n$, then $p|a_i$ for some
$i$.

{\bf Proof:}
It is enough to show that if $p|ab$ then $p|a$ or $p|b$.  Let
$g=\gcd(a,p)$.  Then $g|p$, and since $p$ is a prime, there are two
cases.  

Case~1, $g=p$, then since $g|a$, $p|a$.  

Case~2, $g=1$, so
there exist $u,v$ such that $au+pv=1$,
so $abu+pbv=b$.  

Since $p|ab$, and
$p|p$, it follows that $p|(abu+pbv)$, so $p|b$.
\end{frame}

\begin{frame}

{\bf Fundamental Theorem of Arithmetic}

For $a\ge 2$, $a=p_1^{e_1}p_2^{e_2}\cdots p_r^{e_r}$, where $p_i$ are
prime numbers, and other than rearranging primes, this factorization
is unique.

{\bf Proof:}
We first show the existence of the factorization, and then its
uniqueness.  

The proof of existence is by complete induction; the
basis case is $a=2$, where $2$ is a prime.  

Consider an integer $a>2$;
if $a$ is prime then it is its own factorization (just as in the basis
case).  

Otherwise, if $a$ is composite, then $a=b\cdot c$, where
$1<b,c<a$; apply the induction hypothesis to $b$ and $c$.  

\end{frame}

\begin{frame}

To show uniqueness suppose that $a=p_1p_2\ldots p_s=q_1q_2\ldots q_t$
where we have written out all the primes, that is, instead of writing
$p^e$ we write $p\cdot p\cdots p$, $e$ times.  

Since $p_1|a$, it
follows that
$p_1|q_1q_2\ldots q_t$.  So $p_1|q_j$ for some $j$,
but then $p_1=q_j$ since they are both primes.

Now delete $p_1$ from the first list and $q_j$ from the second list,
and continue.  

Obviously we cannot end up with a product of primes
equal to~1, so the two list must be identical. 
\end{frame}

\begin{frame}

Let $m\ge 1$ be an integer.  We say that $a$ and $b$ are 
\df{congruent modulo $m$}, and
write $a\equiv b\pmod m$ (or sometimes
$a\equiv_mb$) if $m|(a-b)$.  


Another way to say this is that $a$ and
$b$ have the same remainder when divided by $m$;
we can say that $a\equiv
b\pmod m$ if and only if $\rem(m,a)=\rem(m,b)$.

{\bf Facts:} $a_1\equiv_ma_2$ and $b_1\equiv_mb_2$, then $a_1\pm
b_1\equiv_m a_2\pm b_2$ and $a_1\cdot b_1\equiv_m a_2\cdot b_2$.

\end{frame}

\begin{frame}

{\bf Proposition:}
If $m\ge 1$, then $a\cdot b\equiv_m1$ for some $b$ if and only if
$\gcd(a,m)=1$.

{\bf Proof:}
($\Rightarrow$) If there exists a $b$ such that $a\cdot b\equiv_m1$,
then we have $m|(ab-1)$ and so there exists a $c$ such that $ab-1=cm$,
i.e., $ab-cm=1$.  

And since $\gcd(a,m)$ divides both $a$ and $m$, it
also divides $ab-cm$, and so $\gcd(a,m)|1$ and so it must be equal to
$1$.

($\Leftarrow$)  Suppose that $\gcd(a,m)=1$.  By the extended
Euclid's algorithm
there exist $u,v$ such that $au+mv=1$,
so
$au-1=-mv$, so $m|(au-1)$, so
$au\equiv_m1$.  So let $b=u$.

\end{frame}

\begin{frame}

Let $\mathbb{Z}_m=\{0,1,2,\ldots,m-1\}$.  

We call $\mathbb{Z}_m$ the
set of integers modulo $m$.  

To add or multiply in the set
$\mathbb{Z}_m$, we add and multiply the corresponding integers, and
then take the reminder of the division by $m$ as the result.  

Let
$\mathbb{Z}_m^*=\{a\in\mathbb{Z}_m|\gcd(a,m)=1\}$. 

$\mathbb{Z}_m^*$ is the subset
of $\mathbb{Z}_m$ consisting of those elements which have
multiplicative inverses in $\mathbb{Z}_m$.

\end{frame}

\begin{frame}

The function $\phi(n)$ is called the \df{Euler totient
function}, and
it is the number of elements less than $n$ that are co-prime to $n$,
i.e., $\phi(n)=|\mathbb{Z}_n^*|$.   

If we are able to factor, we are
also able to compute $\phi(n)$: suppose that
$n=p_1^{k_1}p_2^{k_2}\cdots p_l^{k_l}$, then it is not hard to see
that $\phi(n)=\prod_{i=1}^lp_i^{k_i-1}(p_i-1)$.

\end{frame}

\begin{frame}

{\bf Fermat's Little Theorem}
Let $p$ be a prime number and $\gcd(a,p)=1$.  Then $a^{p-1}\equiv
1\pmod p$.

{\bf Proof:}
For any $a$ such that $\gcd(a,p)=1$ the following products
\begin{equation}\label{eq:list}
1a,2a,3a,\ldots,(p-1)a,
\end{equation}
all taken mod $p$, are pairwise distinct.  

To see this suppose that
$ja\equiv ka\pmod p$.  Then $(j-k)a\equiv 0\pmod p$, and so
$p|(j-k)a$.  

But since by assumption $\gcd(a,p)=1$, it follows that
$p\not|a$, and so it must be the case that
$p|(j-k)$.  

But since $j,k\in\{1,2,\ldots,p-1\}$, it follows that
$-(p-2)\le j-k\le (p-2)$, so $j-k=0$, i.e., $j=k$.

\end{frame}

\begin{frame}

Thus the numbers in the list~(\ref{eq:list}) are just a reordering of
the list $\{1,2,\ldots,p-1\}$.  

Therefore
\begin{equation}\label{eq:sides}
a^{p-1}(p-1)!\equiv_p
\prod_{j=1}^{p-1}j\cdot a\equiv_p\prod_{j=1}^{p-1}j\equiv_p(p-1)!.
\end{equation}
Since all the numbers in $\{1,2,\ldots,p-1\}$ have inverses in
$\mathbb{Z}_p$, as $\gcd(i,p)=1$ for $1\le i\le p-1$, their product
also has an inverse.  

That is, $(p-1)!$ has an inverse, and so
multiplying both sides of~(\ref{eq:sides}) by $((p-1)!)^{-1}$ we
obtain the result.

\end{frame}

\begin{frame}

{\bf Exercise:}
Give a second proof of Fermat's Little theorem
using the binomial
expansion, i.e., $(x+y)^n=\sum_{j=0}^n{\binom{n}{j}}x^jy^{n-j}$
applied
to $(a+1)^p$.
\end{frame}

\section{Group theory}

\begin{frame}
\frametitle{Group theory}

We say that $(G,\ast)$ is a \df{group} if $G$ is
a set
and $\ast$ is an operation, such that if $a,b\in G$, then $a\ast b\in
G$; this property is called \df{closure}.

The operation $\ast$ has to satisfy the following 3
properties:
\begin{enumerate}
\item  \df{identity law:}
There exists an $e\in G$ such that $e\ast
a=a\ast e=a$ for all $a\in G$.
\item  \df{inverse law:}
For every $a\in G$ there exists an element
$b\in G$ such that $a\ast b=b\ast a=e$.  This element $b$ is called an
{\em inverse} and it can be shown that it is unique; hence it is often
denoted as $a^{-1}$.
\item  \df{associative law:}
For all $a,b,c\in G$, we have
$a\ast(b\ast c)=(a\ast b)\ast c$.
\end{enumerate}
If $(G,\ast)$ also satisfies the \df{commutative law}, that is, if for
all $a,b\in G$, $a\ast b=b\ast a$, then it is called a
\df{commutative} or \df{Abelian}.
\end{frame}

\begin{frame}

Typical examples of groups are
$(\mathbb{Z}_n,+)$\index{group!$(\mathbb{Z}_n,+)$} (integers mod $n$
under addition)

$(\mathbb{Z}_n^*,\cdot)$\index{group!$(\mathbb{Z}_n^*,\cdot)$} 
(integers mod $n$ under
multiplication).  

Note that both these groups are Abelian.  

These are,
of course, the two groups of concern for us; but there are many
others: $(\mathbb{Q},+)$
is an infinite group (rationals under
addition),

$\mathrm{GL}(n,\mathbb{F})$
(which is the group of $n\times n$
invertible matrices over a field $\mathbb{F}$), 

and $S_n$ (the
{\em symmetric group}\index{group!symmetric} over $n$ elements,
consisting of permutations of $[n]$ where $\ast$ is function
composition).

\end{frame}

\begin{frame}

{\bf Exercise:}
Show that $(\mathbb{Z}_n,+)$ and $(\mathbb{Z}_n^*,\cdot)$ are groups,
by checking that the corresponding operation satisfies the three
axioms of a group.

\end{frame}

\begin{frame}

We let $|G|$ denote the number of elements in $G$ (note that $G$ may
be infinite, but we are concerned mainly with finite groups).  

If $g\in G$ and $x\in\mathbb{N}$, then $g^x=g\ast g\ast\cdots\ast g$, $x$
times.  

If it is clear from the context that the operation is $\ast$,
we use juxtaposition $ab$ instead of $a\ast b$.


Suppose that $G$ is a finite group and $a\in G$; then the smallest
$d\in\mathbb{N}$ such that $a^d=e$ is called the 
\df{order} of $a$,
and it is denoted as $\order_G(a)$ (or just $\order(a)$ if the group
$G$ is clear from the context).

\end{frame}

\begin{frame}

{\bf Proposition:}
If $G$ is a finite group, then for all $a\in G$ there exists a
$d\in\mathbb{N}$ such that $a^d=e$.  If $d=\order_G(a)$, and
$a^k=e$, then $d|k$.


{\bf Proof:}
Consider the list $a^1,a^2,a^3,\ldots$.  

If $G$ is finite there must
exist $i<j$ such that $a^i=a^j$.  

Then, $(a^{-1})^i$ applied to both
sides yields $a^{i-j}=e$.  

Let $d=\order(a)$ (by the LNP we know that
it must exist!).  

Suppose that $k\ge d$, $a^k=e$.  Write $k=dq+r$
where $0\le r<d$.  

Then $e=a^k=a^{dq+r}=(a^d)^qa^r=a^r$.  

Since
$a^d=e$ it follows that $a^r=e$, contradicting the minimality of
$d=\order(a)$, unless $r=0$.

\end{frame}

\begin{frame}

If $(G,\ast)$ is a group we say that $H$ is a \df{subgroup}
of $G$,
and write $H\le G$, if $H\subseteq G$ and $H$ is closed under $\ast$.

That is, $H$ is a subset of $G$, and $H$ is itself a group.  

Note that
for any $G$ it is always the case that $\{e\}\le G$ and $G\le G$;
these two are called the \df{trivial subgroups} of $G$.  

If $H\le G$
and $g\in G$, then $gH$ is called a \df{left coset of
$G$}, and it is
simply the set $\{gh|h\in H\}$.  

Note that $gH$ is not necessarily a
subgroup of $G$.

\end{frame}

\begin{frame}

{\bf Lagrange}
If $G$ is a finite group and $H\le G$, then $|H|$ divides $|G|$,
i.e., the order of $H$ divides the order of $G$.

{\bf Proof:}
If $g_1,g_2\in G$, then the two cosets $g_1H$ and $g_2H$ are either
identical or $g_1H\cap g_2H=\emptyset$.  

To see this, suppose that
$g\in g_1H\cap g_2H$, so $g=g_1h_1=g_2h_2$.  

In particular,
$g_1=g_2h_2h_1^{-1}$.  

Thus, $g_1H=(g_2h_2h_1^{-1})H$, and since it
can be easily checked that $(ab)H=a(bH)$ and that $hH=H$ for any $h\in
H$, it follows that $g_1H=g_2H$.

Therefore, for a finite $G=\{g_1,g_2,\ldots,g_n\}$, the collection of
sets $\{g_1H,g_2H,\ldots,g_nH\}$ is a partition of $G$ into subsets
that are either disjoint or identical; from among all subcollections
of identical cosets we pick a representative, so that $G=g_{i_1}H\cup
g_{i_2}H\cup\cdots\cup g_{i_m}H$, and so $|G|=m|H|$, and we are done.
\end{frame}

\begin{frame}

{\bf Exercise:}
Let $H\le G$.  Show that if $h\in H$, then $hH=H$, and that in general
for any $g\in G$, $|gH|=|H|$.  Finally, show that $(ab)H=a(bH)$.  

{\bf Exercise:}
If $G$ is a group, and $\{g_1,g_2,\ldots,g_k\}\subseteq G$, then the
set $\langle g_1,g_2,\ldots,g_k\rangle$ is defined as follows
$$
\{x_1x_2\cdots x_p|p\in\mathbb{N},
x_i\in\{g_1,g_2,\ldots,g_k,g_1^{-1},g_2^{-1},\ldots,g_k^{-1}\}\}.
$$
Show that $\langle g_1,g_2,\ldots,g_k\rangle\le G$, and it is
called the subgroup {\em generated} by $\{g_1,g_2,\ldots,g_k\}$.  Also
show that when $G$ is finite $|\langle g\rangle|=\order_G(g)$.
\end{frame}

\begin{frame}

An example of ``reification.''

{\bf Euler:}
For every $n$ and every $a\in\mathbb{Z}_n^*$, that is, for every pair
$a,n$ such that $\gcd(a,n)=1$, we have
$a^{\phi(n)}\equiv 1\pmod n$.  

{\bf Proof:}
First it is easy to check that $(\mathbb{Z}_n^*,\cdot)$ is a group.

Then by definition $\phi(n)=|\mathbb{Z}_n^*|$, and since $\langle
a\rangle\le\mathbb{Z}_n^*$, it follows by Lagrange's theorem that
$\order(a)=|\langle a\rangle|$ divides $\phi(n)$.

Note that Fermat's Little theorem
is an immediate consequence of
Euler's theorem, since when $p$ is a prime,
$\mathbb{Z}_p^*=\mathbb{Z}_p-\{0\}$, and $\phi(p)=(p-1)$.

\end{frame}

\begin{frame}

{\bf Chinese Remainder}
Given two sets of numbers of equal size,
$r_0,r_1,\ldots,r_n$, and $m_0,m_1,\ldots,m_n$, such that
\begin{equation}\label{eq:rem}
0\le r_i<m_i \qquad 0\le i\le n
\end{equation}
and $\gcd(m_i,m_j)=1$ for $i\neq j$, then there exists an $r$ such
that $r\equiv r_i\pmod{m_i}$ for $0\le i\le n$.

\end{frame}

\begin{frame}

{\bf Proof:}
The proof we give is by counting; we show that the distinct values of
$r$, $0\le r<\Pi m_i$, represent distinct sequences.   

To see that,
note that if $r\equiv r'\pmod{m_i}$ for all $i$, then $m_i|(r-r')$ for
all $i$, and so $(\Pi m_i)|(r-r')$, since the $m_i$'s are pairwise
co-prime.  

So $r\equiv r'\pmod{(\Pi m_i)}$, and so $r=r'$ since both
$r,r'\in\{0,1,\ldots,(\Pi m_i)-1\}$.

But the total number of sequences $r_0,\ldots,r_n$ such
that~(\ref{eq:rem}) holds is precisely $\Pi m_i$.  

Hence every such sequence must be a sequence of remainders of some
$r$, $0\le r<\Pi m_i$.
\end{frame}

\begin{frame}

{\bf Exercise}
The proof of CRT just given is non-constructive.
Show how to obtain efficiently the $r$ that meets the requirement of
the theorem, i.e., in polytime in $n$---so in particular not using
brute force search.

\end{frame}

\begin{frame}

Given two groups $(G_1,\ast_1)$ and $(G_2,\ast_2)$, a mapping
$h:G_1\longrightarrow G_2$ is a \df{homomorphism} if it respects the
operation of the groups; formally, for all $g_1,g_1'\in G_1$,
$h(g_1\ast_1g_1')=h(g_1)\ast_2h(g_1')$.  

If the homomorphism $h$ is
also a bijection, then it is called an \df{isomorphism}.  

If there
exists an isomorphism between two groups $G_1$ and $G_2$, we call them
{\em isomorphic}, and write $G_1\cong G_2$.

If $(G_1,\ast_1)$ and $(G_2,\ast_2)$ are two groups, then their
product, denoted $(G_1\times G_2,\ast)$ is simply $\{(g_1,g_2):g_1\in
G_1,g_2\in G_2\}$, where $(g_1,g_2)\ast(g_1',g_2')$ is
$(g_1\ast_1g_1',g_2\ast_2g_2')$.  

The product of $n$ groups,
$G_1\times G_2\times\cdots\times G_n$ can be defined analogously;
using this notation, the CRT can be stated in the language of group
theory as follows:

If $m_0,m_1,\ldots,m_n$ are pairwise co-prime integers, then
$\mathbb{Z}_{m_0\cdot m_1\cdot\ldots\cdot m_n}\cong
\mathbb{Z}_{m_0}\times
\mathbb{Z}_{m_1}\times\cdots\times
\mathbb{Z}_{m_n}$.
\end{frame}

\end{document}
